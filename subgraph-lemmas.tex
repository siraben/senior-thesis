\begin{coqdoccode}
\coqdocnoindent
\coqdockw{Require} \coqdockw{Import} \coqdocvar{graph}.\coqdoceol
\coqdocnoindent
\coqdockw{Require} \coqdockw{Import} \coqdocvar{List}.\coqdoceol
\coqdocnoindent
\coqdockw{Require} \coqdockw{Import} \coqdocvar{Setoid}. \coqdocnoindent
\coqdockw{Require} \coqdockw{Import} \coqdocvar{FSets}. \coqdocnoindent
\coqdockw{Require} \coqdockw{Import} \coqdocvar{FMaps}. \coqdocnoindent
\coqdockw{Require} \coqdockw{Import} \coqdocvar{PArith}.\coqdoceol
\coqdocnoindent
\coqdockw{Require} \coqdockw{Import} \coqdocvar{Psatz}.\coqdoceol
\coqdocnoindent
\coqdockw{Require} \coqdockw{Import} \coqdocvar{restrict}.\coqdoceol
\coqdocnoindent
\coqdockw{Require} \coqdockw{Import} \coqdocvar{Program}.\coqdoceol
\coqdocnoindent
\coqdockw{Require} \coqdockw{Import} \coqdocvar{FunInd}.\coqdoceol
\coqdocnoindent
\coqdockw{Require} \coqdockw{Import} \coqdocvar{Decidable}.\coqdoceol
\coqdocnoindent
\coqdockw{From} \coqdocvar{Hammer} \coqdockw{Require} \coqdockw{Import} \coqdocvar{Hammer}.\coqdoceol
\coqdocnoindent
\coqdockw{From} \coqdocvar{Hammer} \coqdockw{Require} \coqdockw{Import} \coqdocvar{Tactics}.\coqdoceol
\coqdocnoindent
\coqdockw{Import} \coqdocvar{Arith}.\coqdoceol
\coqdocnoindent
\coqdockw{Import} \coqdocvar{ListNotations}.\coqdoceol
\coqdocnoindent
\coqdockw{Import} \coqdocvar{Nat}.\coqdoceol
\coqdocemptyline
\coqdocnoindent
\coqdockw{Local Open} \coqdockw{Scope} \coqdocvar{nat}.\coqdoceol
\coqdocemptyline
\end{coqdoccode}
\subsection{Properties of subgraphs and degrees}



\subsubsection{Subgraph predicate}


 \coqdocvar{g'} is a subgraph of \coqdocvar{g} if:

\begin{itemize}
\item  the vertex set of \coqdocvar{g'} is a subset of the vertex set of \coqdocvar{g}

\item  the adjacency set of every \coqdocvar{v} in \coqdocvar{g'} is a subset of adjacency set of every \coqdocvar{v} in \coqdocvar{g}


\end{itemize}
\begin{coqdoccode}
\coqdocnoindent
\coqdockw{Definition} \coqdocvar{is\_subgraph} (\coqdocvar{g'} \coqdocvar{g} : \coqdocvar{graph}) :=\coqdoceol
\coqdocindent{1.00em}
\coqdocvar{S.Subset} (\coqdocvar{nodes} \coqdocvar{g'}) (\coqdocvar{nodes} \coqdocvar{g}) \ensuremath{\land} \coqdockw{\ensuremath{\forall}} \coqdocvar{v}, \coqdocvar{S.Subset} (\coqdocvar{adj} \coqdocvar{g'} \coqdocvar{v}) (\coqdocvar{adj} \coqdocvar{g} \coqdocvar{v}).\coqdoceol
\coqdocemptyline
\end{coqdoccode}
\subsubsection{Subgraph relation is reflexive}


\begin{coqdoccode}
\coqdocnoindent
\coqdockw{Lemma} \coqdocvar{subgraph\_refl} : \coqdockw{\ensuremath{\forall}} \coqdocvar{g}, \coqdocvar{is\_subgraph} \coqdocvar{g} \coqdocvar{g}.\coqdoceol
 \coqdocemptyline
\end{coqdoccode}
\subsubsection{Subgraph relation is transitive}


\begin{coqdoccode}
\coqdocemptyline
\coqdocnoindent
\coqdockw{Lemma} \coqdocvar{subgraph\_trans} : \coqdockw{\ensuremath{\forall}} \coqdocvar{g} \coqdocvar{g'} \coqdocvar{g'{}'}, \coqdocvar{is\_subgraph} \coqdocvar{g} \coqdocvar{g'} \ensuremath{\rightarrow} \coqdocvar{is\_subgraph} \coqdocvar{g'} \coqdocvar{g'{}'} \ensuremath{\rightarrow} \coqdocvar{is\_subgraph} \coqdocvar{g} \coqdocvar{g'{}'}.\coqdoceol
 \coqdocemptyline
\end{coqdoccode}
\subsubsection{Subgraphs preserve irrelexivity}


\begin{coqdoccode}
\coqdocnoindent
\coqdockw{Lemma} \coqdocvar{subgraph\_no\_selfloop} : \coqdockw{\ensuremath{\forall}} \coqdocvar{g'} \coqdocvar{g}, \coqdocvar{is\_subgraph} \coqdocvar{g'} \coqdocvar{g} \ensuremath{\rightarrow} \coqdocvar{no\_selfloop} \coqdocvar{g} \ensuremath{\rightarrow} \coqdocvar{no\_selfloop} \coqdocvar{g'}.\coqdoceol
 \coqdocemptyline
\end{coqdoccode}
\subsubsection{Vertices in the subgraph are in original graph}


\begin{coqdoccode}
\coqdocemptyline
\coqdocnoindent
\coqdockw{Lemma} \coqdocvar{subgraph\_vert\_m} : \coqdockw{\ensuremath{\forall}} \coqdocvar{g'} \coqdocvar{g} \coqdocvar{v}, \coqdocvar{is\_subgraph} \coqdocvar{g'} \coqdocvar{g} \ensuremath{\rightarrow} \coqdocvar{M.In} \coqdocvar{v} \coqdocvar{g'} \ensuremath{\rightarrow} \coqdocvar{M.In} \coqdocvar{v} \coqdocvar{g}.\coqdoceol
 \coqdocemptyline
\end{coqdoccode}
\subsubsection{Empty graph is a subgraph}


\begin{coqdoccode}
\coqdocemptyline
\coqdocnoindent
\coqdockw{Lemma} \coqdocvar{empty\_subgraph\_is\_subgraph} (\coqdocvar{g} : \coqdocvar{graph}) : \coqdocvar{is\_subgraph} \coqdocvar{empty\_graph} \coqdocvar{g}.\coqdoceol
\coqdocemptyline
\end{coqdoccode}
\subsection{Induced subgraphs}

\subsubsection{Definition}


\begin{coqdoccode}
\coqdocemptyline
\coqdocnoindent
\coqdockw{Definition} \coqdocvar{subgraph\_of} (\coqdocvar{g} : \coqdocvar{graph}) (\coqdocvar{s} : \coqdocvar{S.t}) :=\coqdoceol
\coqdocindent{1.00em}
\coqdocvar{M.fold} (\coqdockw{fun} \coqdocvar{v} \coqdocvar{adj} \coqdocvar{g'} \ensuremath{\Rightarrow} \coqdockw{if} \coqdocvar{S.mem} \coqdocvar{v} \coqdocvar{s} \coqdockw{then} \coqdocvar{M.add} \coqdocvar{v} (\coqdocvar{S.inter} \coqdocvar{s} \coqdocvar{adj}) \coqdocvar{g'} \coqdockw{else} \coqdocvar{g'}) \coqdocvar{g} \coqdocvar{empty\_graph}.\coqdoceol
\coqdocemptyline
\end{coqdoccode}
\subsubsection{Nodes of an induced subgraph are a subset of the original graph}


\begin{coqdoccode}
\coqdocnoindent
\coqdockw{Lemma} \coqdocvar{subgraph\_vertices} : \coqdockw{\ensuremath{\forall}} \coqdocvar{g} \coqdocvar{s}, \coqdocvar{S.Subset} (\coqdocvar{nodes} (\coqdocvar{subgraph\_of} \coqdocvar{g} \coqdocvar{s})) (\coqdocvar{nodes} \coqdocvar{g}).\coqdoceol
\coqdocemptyline
\end{coqdoccode}
\subsubsection{Edges of an induced subgraph are a subset of the original graph}

 Note that this is defined pointwise: the adjacency set is a subset
    for every vertex.
\begin{coqdoccode}
\coqdocemptyline
\coqdocnoindent
\coqdockw{Lemma} \coqdocvar{subgraph\_edges} : \coqdockw{\ensuremath{\forall}} \coqdocvar{g} \coqdocvar{s} \coqdocvar{v},\coqdoceol
\coqdocindent{2.00em}
\coqdocvar{S.Subset} (\coqdocvar{adj} (\coqdocvar{subgraph\_of} \coqdocvar{g} \coqdocvar{s}) \coqdocvar{v}) (\coqdocvar{adj} \coqdocvar{g} \coqdocvar{v}).\coqdoceol
\coqdocemptyline
\end{coqdoccode}
\subsubsection{Induced subgraph is subgraph}


\begin{coqdoccode}
\coqdocemptyline
\coqdocnoindent
\coqdockw{Lemma} \coqdocvar{subgraph\_of\_is\_subgraph} : \coqdockw{\ensuremath{\forall}} \coqdocvar{g} \coqdocvar{s}, \coqdocvar{is\_subgraph} (\coqdocvar{subgraph\_of} \coqdocvar{g} \coqdocvar{s}) \coqdocvar{g}.\coqdoceol
\coqdocemptyline
\end{coqdoccode}
\subsection{Removal of nodes}

\subsubsection{Removing a distinct vertex from a graph}

 If \coqdocvar{i} and \coqdocvar{j} are distinct vertices then removing \coqdocvar{j} from the
    graph doesn't affect \coqdocvar{i}'s membership.
\begin{coqdoccode}
\coqdocemptyline
\coqdocnoindent
\coqdockw{Lemma} \coqdocvar{remove\_node\_neq} : \coqdockw{\ensuremath{\forall}} \coqdocvar{g} \coqdocvar{i} \coqdocvar{j}, \coqdocvar{i} \ensuremath{\not=} \coqdocvar{j} \ensuremath{\rightarrow} \coqdocvar{M.In} \coqdocvar{i} \coqdocvar{g} \ensuremath{\leftrightarrow} \coqdocvar{M.In} \coqdocvar{i} (\coqdocvar{remove\_node} \coqdocvar{j} \coqdocvar{g}).\coqdoceol
\coqdocemptyline
\end{coqdoccode}
If \coqdocvar{i} is in the graph with \coqdocvar{j} removed then \coqdocvar{i} is not equal to \coqdocvar{j}.
\begin{coqdoccode}
\coqdocemptyline
\coqdocnoindent
\coqdockw{Lemma} \coqdocvar{remove\_node\_neq2} : \coqdockw{\ensuremath{\forall}} \coqdocvar{g} \coqdocvar{i} \coqdocvar{j}, \coqdocvar{M.In} \coqdocvar{i} (\coqdocvar{remove\_node} \coqdocvar{j} \coqdocvar{g}) \ensuremath{\rightarrow} \coqdocvar{i} \ensuremath{\not=} \coqdocvar{j}.\coqdoceol
\coqdocemptyline
\end{coqdoccode}
\subsubsection{Removing a node results in a subgraph}


\begin{coqdoccode}
\coqdocemptyline
\coqdocnoindent
\coqdockw{Lemma} \coqdocvar{remove\_node\_subgraph} : \coqdockw{\ensuremath{\forall}} \coqdocvar{g} \coqdocvar{v}, \coqdocvar{is\_subgraph} (\coqdocvar{remove\_node} \coqdocvar{v} \coqdocvar{g}) \coqdocvar{g}.\coqdoceol
\coqdocemptyline
\end{coqdoccode}
\subsubsection{Removing a node}


\begin{coqdoccode}
\coqdocnoindent
\coqdockw{Lemma} \coqdocvar{remove\_node\_not\_in} : \coqdockw{\ensuremath{\forall}} \coqdocvar{g} \coqdocvar{g'} \coqdocvar{v},\coqdoceol
\coqdocindent{2.00em}
\coqdocvar{is\_subgraph} \coqdocvar{g'} (\coqdocvar{remove\_node} \coqdocvar{v} \coqdocvar{g}) \ensuremath{\rightarrow} \ensuremath{\lnot} \coqdocvar{M.In} \coqdocvar{v} \coqdocvar{g'}.\coqdoceol
\coqdocemptyline
\end{coqdoccode}
\subsubsection{Remove a set of vertices from a graph}

 To make it easier to prove things about it,

\begin{itemize}
\item  first restrict the graph by \coqdocvar{S.diff} (\coqdocvar{Mdomain} \coqdocvar{g}) \coqdocvar{s}

\item  then map subtracting s from every adj set


\end{itemize}
\begin{coqdoccode}
\coqdocnoindent
\coqdockw{Definition} \coqdocvar{remove\_nodes} (\coqdocvar{g} : \coqdocvar{graph}) (\coqdocvar{s} : \coqdocvar{nodeset}) :=\coqdoceol
\coqdocindent{1.00em}
\coqdocvar{M.map} (\coqdockw{fun} \coqdocvar{ve} \ensuremath{\Rightarrow} \coqdocvar{S.diff} \coqdocvar{ve} \coqdocvar{s}) (\coqdocvar{restrict} \coqdocvar{g} (\coqdocvar{S.diff} (\coqdocvar{nodes} \coqdocvar{g}) \coqdocvar{s})).\coqdoceol
\coqdocemptyline
\end{coqdoccode}
\subsubsection{Removing nodes results in a subgraph}


\begin{coqdoccode}
\coqdocnoindent
\coqdockw{Lemma} \coqdocvar{remove\_nodes\_subgraph} : \coqdockw{\ensuremath{\forall}} \coqdocvar{g} \coqdocvar{s}, \coqdocvar{is\_subgraph} (\coqdocvar{remove\_nodes} \coqdocvar{g} \coqdocvar{s}) \coqdocvar{g}.\coqdoceol
\coqdocemptyline
\end{coqdoccode}
\subsubsection{Every vertex in the removing set is not in the resulting graph}


\begin{coqdoccode}
\coqdocemptyline
\coqdocnoindent
\coqdockw{Lemma} \coqdocvar{remove\_nodes\_sub} : \coqdockw{\ensuremath{\forall}} \coqdocvar{g} \coqdocvar{s} \coqdocvar{i}, \coqdocvar{S.In} \coqdocvar{i} \coqdocvar{s} \ensuremath{\rightarrow} \coqdocvar{M.In} \coqdocvar{i} \coqdocvar{g} \ensuremath{\rightarrow} \ensuremath{\lnot} \coqdocvar{M.In} \coqdocvar{i} (\coqdocvar{remove\_nodes} \coqdocvar{g} \coqdocvar{s}).\coqdoceol
\coqdocemptyline
\end{coqdoccode}
\subsubsection{Removing a non-empty set of vertices decreases the size of the graph}


\begin{coqdoccode}
\coqdocemptyline
\coqdocnoindent
\coqdockw{Lemma} \coqdocvar{remove\_nodes\_lt} : \coqdockw{\ensuremath{\forall}} \coqdocvar{g} \coqdocvar{s} \coqdocvar{i}, \coqdocvar{S.In} \coqdocvar{i} \coqdocvar{s} \ensuremath{\rightarrow} \coqdocvar{M.In} \coqdocvar{i} \coqdocvar{g} \ensuremath{\rightarrow} (\coqdocvar{M.cardinal} (\coqdocvar{remove\_nodes} \coqdocvar{g} \coqdocvar{s}) < \coqdocvar{M.cardinal} \coqdocvar{g}).\coqdoceol
\coqdocemptyline
\coqdocnoindent
\coqdockw{Lemma} \coqdocvar{adj\_remove\_nodes\_spec} : \coqdockw{\ensuremath{\forall}} \coqdocvar{g} \coqdocvar{s} \coqdocvar{i} \coqdocvar{j},\coqdoceol
\coqdocindent{2.00em}
\coqdocvar{S.In} \coqdocvar{i} (\coqdocvar{adj} (\coqdocvar{remove\_nodes} \coqdocvar{g} \coqdocvar{s}) \coqdocvar{j}) \ensuremath{\leftrightarrow} \coqdocvar{S.In} \coqdocvar{i} (\coqdocvar{adj} \coqdocvar{g} \coqdocvar{j}) \ensuremath{\land} \ensuremath{\lnot} \coqdocvar{S.In} \coqdocvar{i} \coqdocvar{s} \ensuremath{\land} \ensuremath{\lnot} \coqdocvar{S.In} \coqdocvar{j} \coqdocvar{s}.\coqdoceol
\coqdocemptyline
\coqdocnoindent
\coqdockw{Lemma} \coqdocvar{remove\_nodes\_singleton} : \coqdockw{\ensuremath{\forall}} \coqdocvar{g} \coqdocvar{v}, \coqdocvar{M.Equiv} \coqdocvar{S.Equal} (\coqdocvar{remove\_nodes} \coqdocvar{g} (\coqdocvar{S.singleton} \coqdocvar{v})) (\coqdocvar{remove\_node} \coqdocvar{v} \coqdocvar{g}).\coqdoceol
\coqdocemptyline
\coqdocnoindent
\coqdockw{Lemma} \coqdocvar{remove\_node\_nodes\_adj} : \coqdockw{\ensuremath{\forall}} \coqdocvar{g} \coqdocvar{i} \coqdocvar{v},\coqdoceol
\coqdocindent{2.00em}
\coqdocvar{S.Equal} (\coqdocvar{adj} (\coqdocvar{remove\_nodes} \coqdocvar{g} (\coqdocvar{S.singleton} \coqdocvar{v})) \coqdocvar{i}) (\coqdocvar{adj} (\coqdocvar{remove\_node} \coqdocvar{v} \coqdocvar{g}) \coqdocvar{i}).\coqdoceol
\coqdocemptyline
\coqdocnoindent
\coqdockw{Lemma} \coqdocvar{adj\_remove\_node\_spec} : \coqdockw{\ensuremath{\forall}} \coqdocvar{g} \coqdocvar{v} \coqdocvar{i} \coqdocvar{j},\coqdoceol
\coqdocindent{2.00em}
\coqdocvar{S.In} \coqdocvar{i} (\coqdocvar{adj} (\coqdocvar{remove\_node} \coqdocvar{v} \coqdocvar{g}) \coqdocvar{j}) \ensuremath{\leftrightarrow} \coqdocvar{S.In} \coqdocvar{i} (\coqdocvar{adj} \coqdocvar{g} \coqdocvar{j}) \ensuremath{\land} \coqdocvar{i} \ensuremath{\not=} \coqdocvar{v} \ensuremath{\land} \coqdocvar{j} \ensuremath{\not=} \coqdocvar{v}.\coqdoceol
\coqdocemptyline
\end{coqdoccode}
\subsubsection{Removing a subgraph preserves undirectedness}


\begin{coqdoccode}
\coqdocemptyline
\coqdocnoindent
\coqdockw{Lemma} \coqdocvar{remove\_nodes\_undirected} : \coqdockw{\ensuremath{\forall}} \coqdocvar{g} \coqdocvar{s}, \coqdocvar{undirected} \coqdocvar{g} \ensuremath{\rightarrow} \coqdocvar{undirected} (\coqdocvar{remove\_nodes} \coqdocvar{g} \coqdocvar{s}).\coqdoceol
\coqdocemptyline
\end{coqdoccode}
\subsubsection{Removing a subgraph preserves irreflexivity}


\begin{coqdoccode}
\coqdocemptyline
\coqdocnoindent
\coqdockw{Lemma} \coqdocvar{remove\_nodes\_no\_selfloop} : \coqdockw{\ensuremath{\forall}} \coqdocvar{g} \coqdocvar{s}, \coqdocvar{no\_selfloop} \coqdocvar{g} \ensuremath{\rightarrow} \coqdocvar{no\_selfloop} (\coqdocvar{remove\_nodes} \coqdocvar{g} \coqdocvar{s}).\coqdoceol
\coqdocemptyline
\end{coqdoccode}
\subsubsection{Removing a node preserves undirectedness}


\begin{coqdoccode}
\coqdocemptyline
\coqdocnoindent
\coqdockw{Lemma} \coqdocvar{remove\_node\_undirected} : \coqdockw{\ensuremath{\forall}} \coqdocvar{g} \coqdocvar{i}, \coqdocvar{undirected} \coqdocvar{g} \ensuremath{\rightarrow} \coqdocvar{undirected} (\coqdocvar{remove\_node} \coqdocvar{i} \coqdocvar{g}).\coqdoceol
\coqdocemptyline
\end{coqdoccode}
\subsubsection{Removing a node preserves irreflexivity}


\begin{coqdoccode}
\coqdocemptyline
\coqdocnoindent
\coqdockw{Lemma} \coqdocvar{remove\_node\_no\_selfloop} : \coqdockw{\ensuremath{\forall}} \coqdocvar{g} \coqdocvar{i}, \coqdocvar{no\_selfloop} \coqdocvar{g} \ensuremath{\rightarrow} \coqdocvar{no\_selfloop} (\coqdocvar{remove\_node} \coqdocvar{i} \coqdocvar{g}).\coqdoceol
\coqdocemptyline
\end{coqdoccode}
\subsection{Neighborhood of a vertex}

\subsubsection{Definition of neighbors}


\begin{coqdoccode}
\coqdocemptyline
\coqdocnoindent
\coqdockw{Definition} \coqdocvar{neighbors} (\coqdocvar{g} : \coqdocvar{graph}) \coqdocvar{v} := \coqdocvar{adj} \coqdocvar{g} \coqdocvar{v}.\coqdoceol
\coqdocemptyline
\end{coqdoccode}
\subsubsection{Definition of neighborhood}

 The (open) neighborhood of a vertex v in a graph consists of the
    subgraph induced by the vertices adjacent to v.  It does not
    include v itself.
\begin{coqdoccode}
\coqdocemptyline
\coqdocnoindent
\coqdockw{Definition} \coqdocvar{neighborhood} (\coqdocvar{g} : \coqdocvar{graph}) \coqdocvar{v} := \coqdocvar{remove\_node} \coqdocvar{v} (\coqdocvar{subgraph\_of} \coqdocvar{g} (\coqdocvar{neighbors} \coqdocvar{g} \coqdocvar{v})).\coqdoceol
\coqdocemptyline
\end{coqdoccode}
\subsubsection{Neighborhoods do not include the vertex}


\begin{coqdoccode}
\coqdocemptyline
\coqdocnoindent
\coqdockw{Lemma} \coqdocvar{nbd\_not\_include\_vertex} \coqdocvar{g} \coqdocvar{v} : \coqdocvar{M.find} \coqdocvar{v} (\coqdocvar{neighborhood} \coqdocvar{g} \coqdocvar{v}) = \coqdocvar{None}.\coqdoceol
\coqdocemptyline
\end{coqdoccode}
\subsubsection{Neighborhood is a subgraph}


\begin{coqdoccode}
\coqdocemptyline
\coqdocnoindent
\coqdockw{Lemma} \coqdocvar{nbd\_subgraph} : \coqdockw{\ensuremath{\forall}} \coqdocvar{g} \coqdocvar{i}, \coqdocvar{is\_subgraph} (\coqdocvar{neighborhood} \coqdocvar{g} \coqdocvar{i}) \coqdocvar{g}.\coqdoceol
\coqdocemptyline
\end{coqdoccode}
\subsubsection{Vertices of an induced subgraph are a subset}


\begin{coqdoccode}
\coqdocemptyline
\coqdocnoindent
\coqdockw{Lemma} \coqdocvar{subgraph\_vertices\_set} : \coqdockw{\ensuremath{\forall}} \coqdocvar{g} \coqdocvar{s}, \coqdocvar{S.Subset} (\coqdocvar{nodes} (\coqdocvar{subgraph\_of} \coqdocvar{g} \coqdocvar{s})) \coqdocvar{s}.\coqdoceol
\coqdocemptyline
\end{coqdoccode}
If i is in the induced subgraph then i is in the set of inducing
    vertices.
\begin{coqdoccode}
\coqdocemptyline
\coqdocnoindent
\coqdockw{Lemma} \coqdocvar{subgraph\_of\_nodes} : \coqdockw{\ensuremath{\forall}} \coqdocvar{g} \coqdocvar{i} \coqdocvar{s}, \coqdocvar{S.In} \coqdocvar{i} (\coqdocvar{nodes} (\coqdocvar{subgraph\_of} \coqdocvar{g} \coqdocvar{s})) \ensuremath{\rightarrow} \coqdocvar{S.In} \coqdocvar{i} \coqdocvar{s}.\coqdoceol
\coqdocemptyline
\end{coqdoccode}
\subsubsection{The adjacency set of any vertex of in an induced subgraph is a subset of the vertex set}


\begin{coqdoccode}
\coqdocemptyline
\coqdocnoindent
\coqdockw{Lemma} \coqdocvar{subgraph\_vertices\_adj} : \coqdockw{\ensuremath{\forall}} \coqdocvar{g} \coqdocvar{s} \coqdocvar{i}, \coqdocvar{S.Subset} (\coqdocvar{adj} (\coqdocvar{subgraph\_of} \coqdocvar{g} \coqdocvar{s}) \coqdocvar{i}) \coqdocvar{s}.\coqdoceol
\coqdocemptyline
\end{coqdoccode}
\subsubsection{In neighborhood implies in adjacency set}


\begin{coqdoccode}
\coqdocemptyline
\coqdocnoindent
\coqdockw{Lemma} \coqdocvar{nbd\_adj} : \coqdockw{\ensuremath{\forall}} \coqdocvar{g} \coqdocvar{i} \coqdocvar{j}, \coqdocvar{S.In} \coqdocvar{j} (\coqdocvar{nodes} (\coqdocvar{neighborhood} \coqdocvar{g} \coqdocvar{i})) \ensuremath{\rightarrow} \coqdocvar{S.In} \coqdocvar{j} (\coqdocvar{adj} \coqdocvar{g} \coqdocvar{i}).\coqdoceol
\coqdocemptyline
\end{coqdoccode}
When is an edge in the induced subgraph?

\begin{itemize}
\item  if \coqdocvar{i}, \coqdocvar{j} in \coqdocvar{S} and (\coqdocvar{i},\coqdocvar{j}) in \coqdocvar{G} then (\coqdocvar{i},\coqdocvar{j}) in $G|_S$

\item  if (\coqdocvar{i},\coqdocvar{j}) in $G|_S$ then (\coqdocvar{i},\coqdocvar{j}) in \coqdocvar{G}

\item  if \coqdocvar{v} in $G|_S$ then \coqdocvar{v} in \coqdocvar{S}

\item  if \coqdocvar{v} in \coqdocvar{S} and \coqdocvar{v} in \coqdocvar{G} then \coqdocvar{v} in $G|_S$

\end{itemize}


\subsection{Degrees and maximum degrees}

 Note that this is a partial function because if the vertex is not
    in the graph and we return 0, we can't tell whether it's actually
    in the graph or not. \subsubsection{Degree of a vertex}


\begin{coqdoccode}
\coqdocnoindent
\coqdockw{Definition} \coqdocvar{degree} (\coqdocvar{v} : \coqdocvar{node}) (\coqdocvar{g} : \coqdocvar{graph}) :=\coqdoceol
\coqdocindent{1.00em}
\coqdockw{match} \coqdocvar{M.find} \coqdocvar{v} \coqdocvar{g} \coqdockw{with}\coqdoceol
\coqdocindent{1.00em}
\ensuremath{|} \coqdocvar{None} \ensuremath{\Rightarrow} \coqdocvar{None}\coqdoceol
\coqdocindent{1.00em}
\ensuremath{|} \coqdocvar{Some} \coqdocvar{a} \ensuremath{\Rightarrow} \coqdocvar{Some} (\coqdocvar{S.cardinal} \coqdocvar{a})\coqdoceol
\coqdocindent{1.00em}
\coqdockw{end}.\coqdoceol
\coqdocemptyline
\end{coqdoccode}
\subsubsection{Maximum degree of a graph}


\begin{coqdoccode}
\coqdocnoindent
\coqdockw{Definition} \coqdocvar{max\_deg} (\coqdocvar{g} : \coqdocvar{graph}) := \coqdocvar{list\_max} (\coqdocvar{map} (\coqdockw{fun} \coqdocvar{p} \ensuremath{\Rightarrow} \coqdocvar{S.cardinal} (\coqdocvar{snd} \coqdocvar{p})) (\coqdocvar{M.elements} \coqdocvar{g})).\coqdoceol
\coqdocemptyline
\end{coqdoccode}
\subsubsection{Inversion lemma for degree}


\begin{coqdoccode}
\coqdocemptyline
\coqdocnoindent
\coqdockw{Lemma} \coqdocvar{degree\_gt\_0\_in} (\coqdocvar{g} : \coqdocvar{graph}) (\coqdocvar{v} : \coqdocvar{node}) \coqdocvar{n} :\coqdoceol
\coqdocindent{1.00em}
\coqdocvar{degree} \coqdocvar{v} \coqdocvar{g} = \coqdocvar{Some} \coqdocvar{n} \ensuremath{\rightarrow} \coqdocvar{M.In} \coqdocvar{v} \coqdocvar{g}.\coqdoceol
\coqdocemptyline
\end{coqdoccode}
\subsubsection{The maximum degree of an empty graph is 0}


\begin{coqdoccode}
\coqdocemptyline
\coqdocnoindent
\coqdockw{Lemma} \coqdocvar{max\_deg\_empty} : \coqdocvar{max\_deg} (@\coqdocvar{M.empty} \coqdocvar{\_}) = 0.\coqdoceol
 \coqdocemptyline
\end{coqdoccode}
\subsubsection{Maximum degree bounds the size of all the adjacency sets}


\begin{coqdoccode}
\coqdocemptyline
\coqdocnoindent
\coqdockw{Lemma} \coqdocvar{max\_deg\_max} : \coqdockw{\ensuremath{\forall}} \coqdocvar{g} \coqdocvar{v} \coqdocvar{e}, \coqdocvar{M.find} \coqdocvar{v} \coqdocvar{g} = \coqdocvar{Some} \coqdocvar{e} \ensuremath{\rightarrow} \coqdocvar{S.cardinal} \coqdocvar{e} \ensuremath{\le} \coqdocvar{max\_deg} \coqdocvar{g}.\coqdoceol
\coqdocemptyline
\end{coqdoccode}
\subsubsection{Max degree being 0 implies non-adjacency of all vertices}


\begin{coqdoccode}
\coqdocemptyline
\coqdocnoindent
\coqdockw{Lemma} \coqdocvar{max\_deg\_0\_adj} (\coqdocvar{g} : \coqdocvar{graph}) \coqdocvar{i} \coqdocvar{j} : \coqdocvar{max\_deg} \coqdocvar{g} = 0 \ensuremath{\rightarrow} \ensuremath{\lnot} \coqdocvar{S.In} \coqdocvar{i} (\coqdocvar{adj} \coqdocvar{g} \coqdocvar{j}).\coqdoceol
\coqdocemptyline
\end{coqdoccode}
\subsubsection{Non-zero max degree implies non-empty graph}


\begin{coqdoccode}
\coqdocemptyline
\coqdocnoindent
\coqdockw{Lemma} \coqdocvar{max\_deg\_gt\_not\_empty} (\coqdocvar{g} : \coqdocvar{graph}) : \coqdocvar{max\_deg} \coqdocvar{g} > 0 \ensuremath{\rightarrow} \ensuremath{\lnot} \coqdocvar{M.Empty} \coqdocvar{g}.\coqdoceol
\coqdocemptyline
\end{coqdoccode}
\subsubsection{Removing a node from a graph removes it from adjaceny sets}


\begin{coqdoccode}
\coqdocnoindent
\coqdockw{Lemma} \coqdocvar{remove\_node\_find} :\coqdoceol
\coqdocindent{1.00em}
\coqdockw{\ensuremath{\forall}} (\coqdocvar{g} : \coqdocvar{graph}) (\coqdocvar{i} \coqdocvar{j} : \coqdocvar{node}) (\coqdocvar{e1} : \coqdocvar{nodeset}),\coqdoceol
\coqdocindent{2.00em}
\coqdocvar{i} \ensuremath{\not=} \coqdocvar{j} \ensuremath{\rightarrow}\coqdoceol
\coqdocindent{2.00em}
\coqdocvar{M.find} \coqdocvar{j} \coqdocvar{g} = \coqdocvar{Some} \coqdocvar{e1} \ensuremath{\rightarrow}\coqdoceol
\coqdocindent{2.00em}
\coqdocvar{M.find} \coqdocvar{j} (\coqdocvar{remove\_node} \coqdocvar{i} \coqdocvar{g}) = \coqdocvar{Some} (\coqdocvar{S.remove} \coqdocvar{i} \coqdocvar{e1}).\coqdoceol
\coqdocemptyline
\end{coqdoccode}
\subsubsection{Removing vertex decreases degree of neighbors}


\begin{coqdoccode}
\coqdocemptyline
\coqdocnoindent
\coqdockw{Lemma} \coqdocvar{vertex\_removed\_nbs\_dec} : \coqdockw{\ensuremath{\forall}} (\coqdocvar{g} : \coqdocvar{graph}) (\coqdocvar{i} \coqdocvar{j} : \coqdocvar{node}) \coqdocvar{n},\coqdoceol
\coqdocindent{2.00em}
\coqdocvar{i} \ensuremath{\not=} \coqdocvar{j} \ensuremath{\rightarrow}\coqdoceol
\coqdocindent{2.00em}
\coqdocvar{S.In} \coqdocvar{i} (\coqdocvar{adj} \coqdocvar{g} \coqdocvar{j}) \ensuremath{\rightarrow}\coqdoceol
\coqdocindent{2.00em}
\coqdocvar{degree} \coqdocvar{j} \coqdocvar{g} = \coqdocvar{Some} (\coqdocvar{S} \coqdocvar{n}) \ensuremath{\rightarrow}\coqdoceol
\coqdocindent{2.00em}
\coqdocvar{degree} \coqdocvar{j} (\coqdocvar{remove\_node} \coqdocvar{i} \coqdocvar{g}) = \coqdocvar{Some} \coqdocvar{n}.\coqdoceol
\coqdocemptyline
\end{coqdoccode}
\subsubsection{S.InL and In agree}


\begin{coqdoccode}
\coqdocemptyline
\coqdocnoindent
\coqdockw{Lemma} \coqdocvar{inl\_in} \coqdocvar{i} \coqdocvar{l} : \coqdocvar{S.InL} \coqdocvar{i} \coqdocvar{l} \ensuremath{\leftrightarrow} \coqdocvar{In} \coqdocvar{i} \coqdocvar{l}.\coqdoceol
\coqdocemptyline
\end{coqdoccode}
\subsubsection{Subset respects list inclusion of elements}


\begin{coqdoccode}
\coqdocemptyline
\coqdocnoindent
\coqdockw{Lemma} \coqdocvar{incl\_subset} \coqdocvar{s} \coqdocvar{s'} : \coqdocvar{S.Subset} \coqdocvar{s} \coqdocvar{s'} \ensuremath{\rightarrow} \coqdocvar{incl} (\coqdocvar{S.elements} \coqdocvar{s}) (\coqdocvar{S.elements} \coqdocvar{s'}).\coqdoceol
\coqdocemptyline
\end{coqdoccode}
\subsubsection{Extract a maximum element from a non-empty list}


\begin{coqdoccode}
\coqdocnoindent
\coqdockw{Lemma} \coqdocvar{list\_max\_witness} : \coqdockw{\ensuremath{\forall}} \coqdocvar{l} \coqdocvar{n}, \coqdocvar{l} \ensuremath{\not=} [] \ensuremath{\rightarrow} \coqdocvar{list\_max} \coqdocvar{l} = \coqdocvar{n} \ensuremath{\rightarrow} \{\coqdocvar{x} \ensuremath{|} \coqdocvar{In} \coqdocvar{x} \coqdocvar{l} \ensuremath{\land} \coqdocvar{x} = \coqdocvar{n}\}.\coqdoceol
\coqdocemptyline
\end{coqdoccode}
\subsubsection{Extract a vertex of maximum degree in an non-empty graph}


\begin{coqdoccode}
\coqdocemptyline
\coqdocnoindent
\coqdockw{Lemma} \coqdocvar{max\_degree\_vert} : \coqdockw{\ensuremath{\forall}} \coqdocvar{g} \coqdocvar{n}, \ensuremath{\lnot} \coqdocvar{M.Empty} \coqdocvar{g} \ensuremath{\rightarrow} \coqdocvar{max\_deg} \coqdocvar{g} = \coqdocvar{n} \ensuremath{\rightarrow} \coqdoctac{\ensuremath{\exists}} \coqdocvar{v}, \coqdocvar{degree} \coqdocvar{v} \coqdocvar{g} = \coqdocvar{Some} \coqdocvar{n}.\coqdoceol
\coqdocemptyline
\end{coqdoccode}
\subsubsection{Subgraph relation respects maximum degree}


\begin{coqdoccode}
\coqdocemptyline
\coqdocnoindent
\coqdockw{Lemma} \coqdocvar{max\_deg\_subgraph} : \coqdockw{\ensuremath{\forall}} (\coqdocvar{g} \coqdocvar{g'} : \coqdocvar{graph}), \coqdocvar{is\_subgraph} \coqdocvar{g'} \coqdocvar{g} \ensuremath{\rightarrow} \coqdocvar{max\_deg} \coqdocvar{g'} \ensuremath{\le} \coqdocvar{max\_deg} \coqdocvar{g}.\coqdoceol
\coqdocemptyline
\end{coqdoccode}
\subsubsection{Max degree remains unchanged after removal of non-adjacent max degree vertex}


\begin{coqdoccode}
\coqdocnoindent
\coqdockw{Lemma} \coqdocvar{max\_deg\_remove\_node} :\coqdoceol
\coqdocindent{1.00em}
\coqdockw{\ensuremath{\forall}} (\coqdocvar{n} : \coqdocvar{nat}) (\coqdocvar{g} : \coqdocvar{graph}) (\coqdocvar{v} \coqdocvar{x} : \coqdocvar{node}),\coqdoceol
\coqdocindent{2.00em}
\coqdocvar{max\_deg} \coqdocvar{g} = \coqdocvar{S} \coqdocvar{n} \ensuremath{\rightarrow}\coqdoceol
\coqdocindent{2.00em}
\coqdocvar{degree} \coqdocvar{v} \coqdocvar{g} = \coqdocvar{Some} (\coqdocvar{S} \coqdocvar{n}) \ensuremath{\rightarrow}\coqdoceol
\coqdocindent{2.00em}
\coqdocvar{degree} \coqdocvar{x} \coqdocvar{g} = \coqdocvar{Some} (\coqdocvar{S} \coqdocvar{n}) \ensuremath{\rightarrow}\coqdoceol
\coqdocindent{2.00em}
\ensuremath{\lnot} \coqdocvar{S.In} \coqdocvar{x} (\coqdocvar{adj} \coqdocvar{g} \coqdocvar{v}) \ensuremath{\rightarrow}\coqdoceol
\coqdocindent{2.00em}
\coqdocvar{x} \ensuremath{\not=} \coqdocvar{v} \ensuremath{\rightarrow}\coqdoceol
\coqdocindent{2.00em}
\coqdocvar{max\_deg} (\coqdocvar{remove\_node} \coqdocvar{x} \coqdocvar{g}) = \coqdocvar{S} \coqdocvar{n}.\coqdoceol
\coqdocemptyline
\end{coqdoccode}
\subsection{Vertex extraction}

\subsubsection{Definition for a given degree}


\begin{coqdoccode}
\coqdocemptyline
\coqdocnoindent
\coqdockw{Definition} \coqdocvar{extract\_deg\_vert} (\coqdocvar{g} : \coqdocvar{graph}) (\coqdocvar{d} : \coqdocvar{nat}) :=\coqdoceol
\coqdocindent{1.00em}
\coqdocvar{find} (\coqdockw{fun} \coqdocvar{p} \ensuremath{\Rightarrow} \coqdocvar{Nat.eqb} (\coqdocvar{S.cardinal} (\coqdocvar{snd} \coqdocvar{p})) \coqdocvar{d}) (\coqdocvar{M.elements} \coqdocvar{g}).\coqdoceol
\coqdocemptyline
\coqdocnoindent
\coqdockw{Lemma} \coqdocvar{InA\_in\_iff} \{\coqdocvar{A}\} : \coqdockw{\ensuremath{\forall}} \coqdocvar{p} (\coqdocvar{l} : \coqdocvar{list} (\coqdocvar{M.key} \ensuremath{\times} \coqdocvar{A})), (\coqdocvar{InA} (@\coqdocvar{M.eq\_key\_elt} \coqdocvar{A}) \coqdocvar{p} \coqdocvar{l}) \ensuremath{\leftrightarrow} \coqdocvar{In} \coqdocvar{p} \coqdocvar{l}.\coqdoceol
 \coqdocemptyline
\end{coqdoccode}
\subsubsection{Decidability of extracting a vertex of a given degree}


\begin{coqdoccode}
\coqdocemptyline
\coqdocnoindent
\coqdockw{Lemma} \coqdocvar{extract\_deg\_vert\_dec} : \coqdockw{\ensuremath{\forall}} (\coqdocvar{g} : \coqdocvar{graph}) (\coqdocvar{d} : \coqdocvar{nat}),\coqdoceol
\coqdocindent{2.00em}
\{\coqdocvar{v} \ensuremath{|} \coqdocvar{degree} \coqdocvar{v} \coqdocvar{g} = \coqdocvar{Some} \coqdocvar{d}\} + \ensuremath{\lnot} \coqdoctac{\ensuremath{\exists}} \coqdocvar{v}, \coqdocvar{degree} \coqdocvar{v} \coqdocvar{g} = \coqdocvar{Some} \coqdocvar{d}.\coqdoceol
\coqdocemptyline
\end{coqdoccode}
\subsection{Iterated extraction}

 This subsection concerns functions that extract a list of vertices
    satisfying a degree criterion and incremental removal from the
    graph.

\subsubsection{Extracting a vertex with a given degree iteratively}


\begin{coqdoccode}
\coqdocnoindent
\coqdockw{Function} \coqdocvar{extract\_vertices\_deg} (\coqdocvar{g} : \coqdocvar{graph}) (\coqdocvar{d} : \coqdocvar{nat}) \{\coqdockw{measure} \coqdocvar{M.cardinal} \coqdocvar{g}\} : \coqdocvar{list} (\coqdocvar{node} \ensuremath{\times} \coqdocvar{graph}) \ensuremath{\times} \coqdocvar{graph} :=\coqdoceol
\coqdocindent{1.00em}
\coqdockw{match} \coqdocvar{extract\_deg\_vert\_dec} \coqdocvar{g} \coqdocvar{d} \coqdockw{with}\coqdoceol
\coqdocindent{1.00em}
\ensuremath{|} \coqdocvar{inl} \coqdocvar{v} \ensuremath{\Rightarrow}\coqdoceol
\coqdocindent{3.00em}
\coqdockw{let} \coqdocvar{g'} := \coqdocvar{remove\_node} (`\coqdocvar{v}) \coqdocvar{g} \coqdoctac{in}\coqdoceol
\coqdocindent{3.00em}
\coqdockw{let} (\coqdocvar{l}, \coqdocvar{g'{}'}) := \coqdocvar{extract\_vertices\_deg} \coqdocvar{g'} \coqdocvar{d} \coqdoctac{in}\coqdoceol
\coqdocindent{3.00em}
((`\coqdocvar{v}, \coqdocvar{g'}) :: \coqdocvar{l}, \coqdocvar{g'{}'})\coqdoceol
\coqdocindent{1.00em}
\ensuremath{|} \coqdocvar{inr} \coqdocvar{\_} \ensuremath{\Rightarrow} (\coqdocvar{nil}, \coqdocvar{g})\coqdoceol
\coqdocindent{1.00em}
\coqdockw{end}.\coqdoceol
\coqdocemptyline
\coqdocnoindent
\coqdockw{Functional Scheme} \coqdocvar{extract\_vertices\_deg\_ind} := \coqdockw{Induction} \coqdockw{for} \coqdocvar{extract\_vertices\_deg} \coqdockw{Sort} \coqdockw{Prop}.\coqdoceol
\coqdocemptyline
\coqdocnoindent
\coqdockw{Definition} \coqdocvar{remove\_deg\_n\_graph} \coqdocvar{g} \coqdocvar{n} := \coqdocvar{snd} (\coqdocvar{extract\_vertices\_deg} \coqdocvar{g} \coqdocvar{n}).\coqdoceol
\coqdocnoindent
\coqdockw{Definition} \coqdocvar{remove\_deg\_n\_trace} \coqdocvar{g} \coqdocvar{n} := \coqdocvar{fst} (\coqdocvar{extract\_vertices\_deg} \coqdocvar{g} \coqdocvar{n}).\coqdoceol
\coqdocemptyline
\end{coqdoccode}
\subsubsection{Iterative extraction exhausts vertices of that (non-zero) degree}


\begin{coqdoccode}
\coqdocnoindent
\coqdockw{Lemma} \coqdocvar{extract\_vertices\_deg\_exhaust} (\coqdocvar{g} : \coqdocvar{graph}) \coqdocvar{n} :\coqdoceol
\coqdocindent{1.00em}
\coqdocvar{n} > 0 \ensuremath{\rightarrow} \ensuremath{\lnot} \coqdoctac{\ensuremath{\exists}} \coqdocvar{v}, \coqdocvar{degree} \coqdocvar{v} (\coqdocvar{remove\_deg\_n\_graph} \coqdocvar{g} \coqdocvar{n}) = \coqdocvar{Some} \coqdocvar{n}.\coqdoceol
\coqdocemptyline
\coqdocnoindent
\coqdockw{Lemma} \coqdocvar{mempty\_dec} \{\coqdocvar{A}\} (\coqdocvar{m} : \coqdocvar{M.t} \coqdocvar{A}) : \{\coqdocvar{M.Empty} \coqdocvar{m}\} + \{\~{} \coqdocvar{M.Empty} \coqdocvar{m}\}.\coqdoceol
\coqdocemptyline
\coqdocnoindent
\coqdockw{Lemma} \coqdocvar{extract\_vertices\_deg\_subgraph1} \coqdocvar{g} \coqdocvar{g'} \coqdocvar{g'{}'} \coqdocvar{n} \coqdocvar{v} \coqdocvar{l} :\coqdoceol
\coqdocindent{1.00em}
\coqdocvar{extract\_vertices\_deg} \coqdocvar{g} \coqdocvar{n} = ((\coqdocvar{v}, \coqdocvar{g'}) :: \coqdocvar{l}, \coqdocvar{g'{}'}) \ensuremath{\rightarrow} \coqdocvar{is\_subgraph} \coqdocvar{g'} \coqdocvar{g}.\coqdoceol
\coqdocemptyline
\end{coqdoccode}
\subsection{Subgraph series}

 A subgraph series is a list of subgraphs such that later elements
    are subgraphs of former elements.
\begin{coqdoccode}
\coqdocemptyline
\coqdocnoindent
\coqdockw{Inductive} \coqdocvar{subgraph\_series} : \coqdocvar{list} \coqdocvar{graph} \ensuremath{\rightarrow} \coqdockw{Prop} :=\coqdoceol
\coqdocnoindent
\ensuremath{|} \coqdocvar{sg\_nil} : \coqdocvar{subgraph\_series} []\coqdoceol
\coqdocnoindent
\ensuremath{|} \coqdocvar{sg\_single} : \coqdockw{\ensuremath{\forall}} \coqdocvar{g}, \coqdocvar{subgraph\_series} [\coqdocvar{g}]\coqdoceol
\coqdocnoindent
\ensuremath{|} \coqdocvar{sg\_cons} : \coqdockw{\ensuremath{\forall}} \coqdocvar{g} \coqdocvar{g'} \coqdocvar{l}, \coqdocvar{is\_subgraph} \coqdocvar{g'} \coqdocvar{g} \ensuremath{\rightarrow} \coqdocvar{subgraph\_series} (\coqdocvar{g'} :: \coqdocvar{l}) \ensuremath{\rightarrow} \coqdocvar{subgraph\_series} (\coqdocvar{g} :: \coqdocvar{g'} :: \coqdocvar{l}).\coqdoceol
\coqdocemptyline
\end{coqdoccode}
The subgraphs created by the extraction are a subgraph series
\begin{coqdoccode}
\coqdocemptyline
\coqdocnoindent
\coqdockw{Lemma} \coqdocvar{extract\_vertices\_deg\_series} \coqdocvar{g} \coqdocvar{n} :\coqdoceol
\coqdocindent{1.00em}
\coqdocvar{subgraph\_series} (\coqdocvar{map} \coqdocvar{snd} (\coqdocvar{remove\_deg\_n\_trace} \coqdocvar{g} \coqdocvar{n})).\coqdoceol
\coqdocemptyline
\end{coqdoccode}
\subsubsection{The final graph returned by the vertex extraction is a subgraph.}


\begin{coqdoccode}
\coqdocemptyline
\coqdocnoindent
\coqdockw{Lemma} \coqdocvar{extract\_vertices\_deg\_subgraph} (\coqdocvar{g} : \coqdocvar{graph}) \coqdocvar{n} :\coqdoceol
\coqdocindent{1.00em}
\coqdocvar{is\_subgraph} (\coqdocvar{remove\_deg\_n\_graph} \coqdocvar{g} \coqdocvar{n}) \coqdocvar{g}.\coqdoceol
\coqdocemptyline
\end{coqdoccode}
\subsubsection{Max degree 0 implies all vertices have degree 0}


\begin{coqdoccode}
\coqdocemptyline
\coqdocnoindent
\coqdockw{Lemma} \coqdocvar{max\_deg\_0\_all\_0} : \coqdockw{\ensuremath{\forall}} (\coqdocvar{g} : \coqdocvar{graph}) \coqdocvar{v}, \coqdocvar{max\_deg} \coqdocvar{g} = 0 \ensuremath{\rightarrow} \coqdocvar{M.In} \coqdocvar{v} \coqdocvar{g} \ensuremath{\rightarrow} \coqdocvar{degree} \coqdocvar{v} \coqdocvar{g} = \coqdocvar{Some} 0.\coqdoceol
\coqdocemptyline
\end{coqdoccode}
\subsubsection{Extracting degree 0 vertices from a max degree 0 graph empties it}


\begin{coqdoccode}
\coqdocemptyline
\coqdocnoindent
\coqdockw{Lemma} \coqdocvar{extract\_vertices\_deg0\_empty} : \coqdockw{\ensuremath{\forall}} (\coqdocvar{g} : \coqdocvar{graph}),\coqdoceol
\coqdocindent{1.00em}
\coqdocvar{max\_deg} \coqdocvar{g} = 0 \ensuremath{\rightarrow} \coqdocvar{M.Empty} (\coqdocvar{remove\_deg\_n\_graph} \coqdocvar{g} 0).\coqdoceol
\coqdocemptyline
\end{coqdoccode}
\subsubsection{Extracting all max degree vertices strictly decreases max degree}


\begin{coqdoccode}
\coqdocemptyline
\coqdocnoindent
\coqdockw{Lemma} \coqdocvar{extract\_vertices\_max\_deg} (\coqdocvar{g} : \coqdocvar{graph}) :\coqdoceol
\coqdocindent{1.50em}
\coqdocvar{max\_deg} \coqdocvar{g} > 0 \ensuremath{\rightarrow} \coqdocvar{max\_deg} (\coqdocvar{remove\_deg\_n\_graph} \coqdocvar{g} (\coqdocvar{max\_deg} \coqdocvar{g})) < \coqdocvar{max\_deg} \coqdocvar{g}.\coqdoceol
\coqdocemptyline
\end{coqdoccode}
\subsubsection{Subgraph respects degree of vertices}


\begin{coqdoccode}
\coqdocemptyline
\coqdocnoindent
\coqdockw{Lemma} \coqdocvar{degree\_subgraph} (\coqdocvar{g} \coqdocvar{g'}: \coqdocvar{graph}) \coqdocvar{v} \coqdocvar{n} \coqdocvar{m} :\coqdoceol
\coqdocindent{1.00em}
\coqdocvar{is\_subgraph} \coqdocvar{g} \coqdocvar{g'} \ensuremath{\rightarrow} \coqdocvar{degree} \coqdocvar{v} \coqdocvar{g} = \coqdocvar{Some} \coqdocvar{n} \ensuremath{\rightarrow} \coqdocvar{degree} \coqdocvar{v} \coqdocvar{g'} = \coqdocvar{Some} \coqdocvar{m} \ensuremath{\rightarrow} \coqdocvar{n} \ensuremath{\le} \coqdocvar{m}.\coqdoceol
\coqdocemptyline
\end{coqdoccode}
\subsubsection{Degree of a node that is removed is 0}


\begin{coqdoccode}
\coqdocnoindent
\coqdockw{Lemma} \coqdocvar{degree\_remove} (\coqdocvar{g} : \coqdocvar{graph}) \coqdocvar{v} :\coqdoceol
\coqdocindent{1.00em}
\coqdocvar{degree} \coqdocvar{v} (\coqdocvar{remove\_node} \coqdocvar{v} \coqdocvar{g}) = \coqdocvar{None}.\coqdoceol
\coqdocemptyline
\end{coqdoccode}
\subsubsection{Maximum degree in a subgraph implies maximum degree in original}


\begin{coqdoccode}
\coqdocnoindent
\coqdockw{Lemma} \coqdocvar{max\_deg\_subgraph\_inv} : \coqdockw{\ensuremath{\forall}} (\coqdocvar{g'} \coqdocvar{g} : \coqdocvar{graph}) \coqdocvar{v},\coqdoceol
\coqdocindent{2.00em}
\coqdocvar{is\_subgraph} \coqdocvar{g'} \coqdocvar{g} \ensuremath{\rightarrow}\coqdoceol
\coqdocindent{2.00em}
\coqdocvar{degree} \coqdocvar{v} \coqdocvar{g'} = \coqdocvar{Some} (\coqdocvar{max\_deg} \coqdocvar{g}) \ensuremath{\rightarrow}\coqdoceol
\coqdocindent{2.00em}
\coqdocvar{degree} \coqdocvar{v} \coqdocvar{g} = \coqdocvar{Some} (\coqdocvar{max\_deg} \coqdocvar{g}).\coqdoceol
\coqdocemptyline
\end{coqdoccode}
\subsubsection{Non-adjacency of max degree vertices after one step}

 If a vertex of max degree is removed from a graph then any vertex
    with max degree in the new graph cannot be adjacent to it.
\begin{coqdoccode}
\coqdocemptyline
\coqdocnoindent
\coqdockw{Lemma} \coqdocvar{remove\_max\_deg\_adj} : \coqdockw{\ensuremath{\forall}} (\coqdocvar{g} : \coqdocvar{graph}) (\coqdocvar{i} \coqdocvar{j} : \coqdocvar{node}),\coqdoceol
\coqdocindent{2.00em}
\coqdocvar{i} \ensuremath{\not=} \coqdocvar{j} \ensuremath{\rightarrow}\coqdoceol
\coqdocindent{2.00em}
\coqdocvar{M.In} \coqdocvar{i} \coqdocvar{g} \ensuremath{\rightarrow}\coqdoceol
\coqdocindent{2.00em}
\coqdocvar{degree} \coqdocvar{j} \coqdocvar{g} = \coqdocvar{Some} (\coqdocvar{max\_deg} \coqdocvar{g}) \ensuremath{\rightarrow}\coqdoceol
\coqdocindent{2.00em}
\coqdocvar{degree} \coqdocvar{j} (\coqdocvar{remove\_node} \coqdocvar{i} \coqdocvar{g}) = \coqdocvar{Some} (\coqdocvar{max\_deg} \coqdocvar{g}) \ensuremath{\rightarrow}\coqdoceol
\coqdocindent{2.00em}
\ensuremath{\lnot} (\coqdocvar{S.In} \coqdocvar{i} (\coqdocvar{adj} \coqdocvar{g} \coqdocvar{j})).\coqdoceol
\coqdocemptyline
\end{coqdoccode}
The same as above, but for non-zero degree graphs
\begin{coqdoccode}
\coqdocemptyline
\coqdocnoindent
\coqdockw{Lemma} \coqdocvar{remove\_max\_deg\_adj'} : \coqdockw{\ensuremath{\forall}} (\coqdocvar{g} : \coqdocvar{graph}) (\coqdocvar{i} \coqdocvar{j} : \coqdocvar{node}),\coqdoceol
\coqdocindent{2.00em}
\coqdocvar{max\_deg} \coqdocvar{g} > 0 \ensuremath{\rightarrow}\coqdoceol
\coqdocindent{2.00em}
\coqdocvar{M.In} \coqdocvar{i} \coqdocvar{g} \ensuremath{\rightarrow}\coqdoceol
\coqdocindent{2.00em}
\coqdocvar{degree} \coqdocvar{j} \coqdocvar{g} = \coqdocvar{Some} (\coqdocvar{max\_deg} \coqdocvar{g}) \ensuremath{\rightarrow}\coqdoceol
\coqdocindent{2.00em}
\coqdocvar{degree} \coqdocvar{j} (\coqdocvar{remove\_node} \coqdocvar{i} \coqdocvar{g}) = \coqdocvar{Some} (\coqdocvar{max\_deg} \coqdocvar{g}) \ensuremath{\rightarrow}\coqdoceol
\coqdocindent{2.00em}
\ensuremath{\lnot} (\coqdocvar{S.In} \coqdocvar{i} (\coqdocvar{adj} \coqdocvar{g} \coqdocvar{j})).\coqdoceol
\coqdocemptyline
\coqdocnoindent
\coqdockw{Lemma} \coqdocvar{not\_adj\_remove} : \coqdockw{\ensuremath{\forall}} (\coqdocvar{g} : \coqdocvar{graph}) (\coqdocvar{n} \coqdocvar{m} \coqdocvar{p} : \coqdocvar{node}),\coqdoceol
\coqdocindent{2.00em}
\coqdocvar{n} \ensuremath{\not=} \coqdocvar{m} \ensuremath{\rightarrow} \coqdocvar{m} \ensuremath{\not=} \coqdocvar{p} \ensuremath{\rightarrow}\coqdoceol
\coqdocindent{2.00em}
\ensuremath{\lnot} (\coqdocvar{S.In} \coqdocvar{n} (\coqdocvar{adj} (\coqdocvar{remove\_node} \coqdocvar{m} \coqdocvar{g}) \coqdocvar{p})) \ensuremath{\rightarrow}\coqdoceol
\coqdocindent{2.00em}
\ensuremath{\lnot} (\coqdocvar{S.In} \coqdocvar{n} (\coqdocvar{adj} \coqdocvar{g} \coqdocvar{p})).\coqdoceol
\coqdocemptyline
\coqdocnoindent
\coqdockw{Lemma} \coqdocvar{not\_adj\_removes} : \coqdockw{\ensuremath{\forall}} (\coqdocvar{g} : \coqdocvar{graph}) (\coqdocvar{n} \coqdocvar{p} : \coqdocvar{node}) \coqdocvar{s},\coqdoceol
\coqdocindent{2.00em}
\ensuremath{\lnot} \coqdocvar{S.In} \coqdocvar{n} \coqdocvar{s} \ensuremath{\rightarrow} \ensuremath{\lnot} \coqdocvar{S.In} \coqdocvar{p} \coqdocvar{s} \ensuremath{\rightarrow}\coqdoceol
\coqdocindent{2.00em}
\ensuremath{\lnot} (\coqdocvar{S.In} \coqdocvar{n} (\coqdocvar{adj} (\coqdocvar{remove\_nodes} \coqdocvar{g} \coqdocvar{s}) \coqdocvar{p})) \ensuremath{\rightarrow}\coqdoceol
\coqdocindent{2.00em}
\ensuremath{\lnot} (\coqdocvar{S.In} \coqdocvar{n} (\coqdocvar{adj} \coqdocvar{g} \coqdocvar{p})).\coqdoceol
\coqdocemptyline
\end{coqdoccode}
\subsection{Independent sets}

 An independent set is a set of vertices such that no two are adjacent.
\begin{coqdoccode}
\coqdocemptyline
\coqdocnoindent
\coqdockw{Definition} \coqdocvar{independent\_set} (\coqdocvar{g} : \coqdocvar{graph}) (\coqdocvar{s} : \coqdocvar{nodeset}) :=\coqdoceol
\coqdocindent{1.00em}
\coqdockw{\ensuremath{\forall}} \coqdocvar{i} \coqdocvar{j}, \coqdocvar{S.In} \coqdocvar{i} \coqdocvar{s} \ensuremath{\rightarrow} \coqdocvar{S.In} \coqdocvar{j} \coqdocvar{s} \ensuremath{\rightarrow} \ensuremath{\lnot} \coqdocvar{S.In} \coqdocvar{i} (\coqdocvar{adj} \coqdocvar{g} \coqdocvar{j}).\coqdoceol
\coqdocemptyline
\coqdocnoindent
\coqdockw{Lemma} \coqdocvar{independent\_set\_add} \coqdocvar{g} \coqdocvar{s} \coqdocvar{i} :\coqdoceol
\coqdocindent{1.00em}
\coqdocvar{no\_selfloop} \coqdocvar{g} \ensuremath{\rightarrow} \coqdocvar{undirected} \coqdocvar{g} \ensuremath{\rightarrow}\coqdoceol
\coqdocindent{1.00em}
(\coqdockw{\ensuremath{\forall}} \coqdocvar{j}, \coqdocvar{S.In} \coqdocvar{j} \coqdocvar{s} \ensuremath{\rightarrow} \ensuremath{\lnot} \coqdocvar{S.In} \coqdocvar{i} (\coqdocvar{adj} \coqdocvar{g} \coqdocvar{j})) \ensuremath{\rightarrow}\coqdoceol
\coqdocindent{1.00em}
\coqdocvar{independent\_set} \coqdocvar{g} \coqdocvar{s} \ensuremath{\rightarrow} \coqdocvar{independent\_set} \coqdocvar{g} (\coqdocvar{S.add} \coqdocvar{i} \coqdocvar{s}).\coqdoceol
\coqdocemptyline
\coqdocnoindent
\coqdockw{Lemma} \coqdocvar{independent\_set\_subgraph} : \coqdockw{\ensuremath{\forall}} (\coqdocvar{g} \coqdocvar{g'} : \coqdocvar{graph}) (\coqdocvar{s} : \coqdocvar{nodeset}),\coqdoceol
\coqdocindent{2.00em}
\coqdocvar{is\_subgraph} \coqdocvar{g'} \coqdocvar{g} \ensuremath{\rightarrow} \coqdocvar{independent\_set} \coqdocvar{g} \coqdocvar{s} \ensuremath{\rightarrow} \coqdocvar{independent\_set} \coqdocvar{g'} \coqdocvar{s}.\coqdoceol
 \coqdocemptyline
\coqdocnoindent
\coqdockw{Function} \coqdocvar{extract\_vertices\_degs} (\coqdocvar{g} : \coqdocvar{graph}) (\coqdocvar{d} : \coqdocvar{nat}) \{\coqdockw{measure} \coqdocvar{M.cardinal} \coqdocvar{g}\} : \coqdocvar{nodeset} \ensuremath{\times} \coqdocvar{graph} :=\coqdoceol
\coqdocindent{1.00em}
\coqdockw{match} \coqdocvar{extract\_deg\_vert\_dec} \coqdocvar{g} \coqdocvar{d} \coqdockw{with}\coqdoceol
\coqdocindent{1.00em}
\ensuremath{|} \coqdocvar{inl} \coqdocvar{v} \ensuremath{\Rightarrow}\coqdoceol
\coqdocindent{3.00em}
\coqdockw{let} \coqdocvar{g'} := \coqdocvar{remove\_node} (`\coqdocvar{v}) \coqdocvar{g} \coqdoctac{in}\coqdoceol
\coqdocindent{3.00em}
\coqdockw{let} (\coqdocvar{s}, \coqdocvar{g'{}'}) := \coqdocvar{extract\_vertices\_degs} \coqdocvar{g'} \coqdocvar{d} \coqdoctac{in}\coqdoceol
\coqdocindent{3.00em}
(\coqdocvar{S.add} (`\coqdocvar{v}) \coqdocvar{s} , \coqdocvar{g'{}'})\coqdoceol
\coqdocindent{1.00em}
\ensuremath{|} \coqdocvar{inr} \coqdocvar{\_} \ensuremath{\Rightarrow} (\coqdocvar{S.empty}, \coqdocvar{g})\coqdoceol
\coqdocindent{1.00em}
\coqdockw{end}.\coqdoceol
\coqdocemptyline
\coqdocnoindent
\coqdockw{Functional Scheme} \coqdocvar{extract\_vertices\_degs\_ind} := \coqdockw{Induction} \coqdockw{for} \coqdocvar{extract\_vertices\_degs} \coqdockw{Sort} \coqdockw{Prop}.\coqdoceol
\coqdocemptyline
\end{coqdoccode}
\subsubsection{Extracting max degree vertices from a strictly lower max degree subgraph is empty}


\begin{coqdoccode}
\coqdocnoindent
\coqdockw{Lemma} \coqdocvar{extract\_vertices\_degs\_empty} :\coqdoceol
\coqdocindent{1.00em}
\coqdockw{\ensuremath{\forall}} (\coqdocvar{g} \coqdocvar{g'} \coqdocvar{g'{}'} : \coqdocvar{graph}) (\coqdocvar{d} : \coqdocvar{nat}) (\coqdocvar{v} : \coqdocvar{node}) \coqdocvar{s},\coqdoceol
\coqdocindent{2.00em}
\coqdocvar{is\_subgraph} \coqdocvar{g'} \coqdocvar{g} \ensuremath{\rightarrow}\coqdoceol
\coqdocindent{2.00em}
\coqdocvar{d} = \coqdocvar{max\_deg} \coqdocvar{g} \ensuremath{\rightarrow}\coqdoceol
\coqdocindent{2.00em}
\coqdocvar{extract\_vertices\_degs} \coqdocvar{g'} \coqdocvar{d} = (\coqdocvar{s}, \coqdocvar{g'{}'}) \ensuremath{\rightarrow}\coqdoceol
\coqdocindent{2.00em}
\coqdocvar{max\_deg} \coqdocvar{g'} < \coqdocvar{max\_deg} \coqdocvar{g} \ensuremath{\rightarrow}\coqdoceol
\coqdocindent{2.00em}
\coqdocvar{degree} \coqdocvar{v} \coqdocvar{g} = \coqdocvar{Some} \coqdocvar{d} \ensuremath{\rightarrow}\coqdoceol
\coqdocindent{2.00em}
\coqdocvar{S.Empty} \coqdocvar{s}.\coqdoceol
\coqdocemptyline
\coqdocnoindent
\coqdockw{Lemma} \coqdocvar{max\_degree\_extraction\_independent\_set0} : \coqdockw{\ensuremath{\forall}} (\coqdocvar{g} : \coqdocvar{graph}) \coqdocvar{d},\coqdoceol
\coqdocindent{2.00em}
\coqdocvar{no\_selfloop} \coqdocvar{g} \ensuremath{\rightarrow}\coqdoceol
\coqdocindent{2.00em}
\coqdocvar{d} = \coqdocvar{max\_deg} \coqdocvar{g} \ensuremath{\rightarrow}\coqdoceol
\coqdocindent{2.00em}
\coqdocvar{d} = 0 \ensuremath{\rightarrow}\coqdoceol
\coqdocindent{2.00em}
\coqdocvar{independent\_set} \coqdocvar{g} (\coqdocvar{fst} (\coqdocvar{extract\_vertices\_degs} \coqdocvar{g} \coqdocvar{d})) \ensuremath{\land}\coqdoceol
\coqdocindent{3.00em}
(\coqdockw{\ensuremath{\forall}} \coqdocvar{k}, \coqdocvar{S.In} \coqdocvar{k} (\coqdocvar{fst} (\coqdocvar{extract\_vertices\_degs} \coqdocvar{g} \coqdocvar{d})) \ensuremath{\rightarrow} \coqdocvar{degree} \coqdocvar{k} \coqdocvar{g} = \coqdocvar{Some} \coqdocvar{d}).\coqdoceol
\coqdocemptyline
\end{coqdoccode}
\subsubsection{Extracting max degree vertices gives an independent set}


\begin{coqdoccode}
\coqdocnoindent
\coqdockw{Lemma} \coqdocvar{max\_degree\_extraction\_independent\_set} : \coqdockw{\ensuremath{\forall}} (\coqdocvar{g} : \coqdocvar{graph}) (\coqdocvar{d} : \coqdocvar{nat}),\coqdoceol
\coqdocindent{2.00em}
\coqdocvar{undirected} \coqdocvar{g} \ensuremath{\rightarrow}\coqdoceol
\coqdocindent{2.00em}
\coqdocvar{no\_selfloop} \coqdocvar{g} \ensuremath{\rightarrow}\coqdoceol
\coqdocindent{2.00em}
\coqdocvar{d} = \coqdocvar{max\_deg} \coqdocvar{g} \ensuremath{\rightarrow}\coqdoceol
\coqdocindent{2.00em}
\coqdocvar{independent\_set} \coqdocvar{g} (\coqdocvar{fst} (\coqdocvar{extract\_vertices\_degs} \coqdocvar{g} \coqdocvar{d})) \ensuremath{\land}\coqdoceol
\coqdocindent{3.00em}
(\coqdockw{\ensuremath{\forall}} \coqdocvar{k}, \coqdocvar{S.In} \coqdocvar{k} (\coqdocvar{fst} (\coqdocvar{extract\_vertices\_degs} \coqdocvar{g} \coqdocvar{d})) \ensuremath{\rightarrow} \coqdocvar{degree} \coqdocvar{k} \coqdocvar{g} = \coqdocvar{Some} \coqdocvar{d}).\coqdoceol
\coqdocemptyline
\end{coqdoccode}
\subsubsection{Iterative extraction exhausts vertices of that (non-zero) degree}


\begin{coqdoccode}
\coqdocnoindent
\coqdockw{Lemma} \coqdocvar{extract\_vertices\_degs\_exhaust} (\coqdocvar{g} \coqdocvar{g'} : \coqdocvar{graph}) \coqdocvar{n} \coqdocvar{ns} :\coqdoceol
\coqdocindent{1.00em}
\coqdocvar{n} > 0 \ensuremath{\rightarrow}\coqdoceol
\coqdocindent{1.00em}
\coqdocvar{extract\_vertices\_degs} \coqdocvar{g} \coqdocvar{n} = (\coqdocvar{ns}, \coqdocvar{g'}) \ensuremath{\rightarrow}\coqdoceol
\coqdocindent{1.00em}
\ensuremath{\lnot} \coqdoctac{\ensuremath{\exists}} \coqdocvar{v}, \coqdocvar{degree} \coqdocvar{v} \coqdocvar{g'} = \coqdocvar{Some} \coqdocvar{n}.\coqdoceol
\coqdocemptyline
\end{coqdoccode}
\subsubsection{Iterative extraction results in a subgraph}


\begin{coqdoccode}
\coqdocnoindent
\coqdockw{Lemma} \coqdocvar{extract\_vertices\_degs\_subgraph} : \coqdockw{\ensuremath{\forall}} (\coqdocvar{g} \coqdocvar{g'} : \coqdocvar{graph}) \coqdocvar{n} \coqdocvar{ns},\coqdoceol
\coqdocindent{1.00em}
\coqdocvar{extract\_vertices\_degs} \coqdocvar{g} \coqdocvar{n} = (\coqdocvar{ns}, \coqdocvar{g'}) \ensuremath{\rightarrow}\coqdoceol
\coqdocindent{1.00em}
\coqdocvar{is\_subgraph} \coqdocvar{g'} \coqdocvar{g}.\coqdoceol
\coqdocemptyline
\end{coqdoccode}
\subsubsection{Iterative extraction preserves undirectedness}


\begin{coqdoccode}
\coqdocnoindent
\coqdockw{Lemma} \coqdocvar{extract\_vertices\_degs\_undirected} : \coqdockw{\ensuremath{\forall}} (\coqdocvar{g} \coqdocvar{g'} : \coqdocvar{graph}) \coqdocvar{n} \coqdocvar{ns},\coqdoceol
\coqdocindent{2.00em}
\coqdocvar{undirected} \coqdocvar{g} \ensuremath{\rightarrow}\coqdoceol
\coqdocindent{2.00em}
\coqdocvar{extract\_vertices\_degs} \coqdocvar{g} \coqdocvar{n} = (\coqdocvar{ns}, \coqdocvar{g'}) \ensuremath{\rightarrow}\coqdoceol
\coqdocindent{2.00em}
\coqdocvar{undirected} \coqdocvar{g'}.\coqdoceol
\coqdocemptyline
\end{coqdoccode}
\subsubsection{Iterative max degree extraction strictly decreases the max degree}


\begin{coqdoccode}
\coqdocnoindent
\coqdockw{Lemma} \coqdocvar{extract\_vertices\_max\_degs} : \coqdockw{\ensuremath{\forall}} (\coqdocvar{g} \coqdocvar{g'} : \coqdocvar{graph}) \coqdocvar{ns},\coqdoceol
\coqdocindent{2.00em}
\coqdocvar{max\_deg} \coqdocvar{g} > 0 \ensuremath{\rightarrow}\coqdoceol
\coqdocindent{2.00em}
\coqdocvar{extract\_vertices\_degs} \coqdocvar{g} (\coqdocvar{max\_deg} \coqdocvar{g}) = (\coqdocvar{ns}, \coqdocvar{g'}) \ensuremath{\rightarrow}\coqdoceol
\coqdocindent{2.00em}
\coqdocvar{max\_deg} \coqdocvar{g'} < \coqdocvar{max\_deg} \coqdocvar{g}.\coqdoceol
\coqdocemptyline
\end{coqdoccode}
\subsubsection{Vertices in extraction are not in resulting graph but are in original graph}

 Later we can use this to show that the vertices after each round
    of extraction are disjoint.
\begin{coqdoccode}
\coqdocemptyline
\coqdocnoindent
\coqdockw{Lemma} \coqdocvar{extract\_vertices\_remove} : \coqdockw{\ensuremath{\forall}} (\coqdocvar{g} \coqdocvar{g'} : \coqdocvar{graph}) \coqdocvar{s} \coqdocvar{n},\coqdoceol
\coqdocindent{2.00em}
(\coqdocvar{s}, \coqdocvar{g'}) = \coqdocvar{extract\_vertices\_degs} \coqdocvar{g} \coqdocvar{n} \ensuremath{\rightarrow}\coqdoceol
\coqdocindent{2.00em}
(\coqdockw{\ensuremath{\forall}} \coqdocvar{v}, \coqdocvar{S.In} \coqdocvar{v} \coqdocvar{s} \ensuremath{\rightarrow} \ensuremath{\lnot} \coqdocvar{M.In} \coqdocvar{v} \coqdocvar{g'} \ensuremath{\land} \coqdocvar{M.In} \coqdocvar{v} \coqdocvar{g}).\coqdoceol
\coqdocemptyline
\end{coqdoccode}
\subsubsection{Disjointness of after each round of extraction}


\begin{coqdoccode}
\coqdocnoindent
\coqdockw{Lemma} \coqdocvar{max\_degree\_extraction\_disjoint} : \coqdockw{\ensuremath{\forall}} (\coqdocvar{g} \coqdocvar{g'} \coqdocvar{g'{}'} : \coqdocvar{graph}) (\coqdocvar{n} \coqdocvar{m} : \coqdocvar{nat}) \coqdocvar{s} \coqdocvar{s'},\coqdoceol
\coqdocindent{2.00em}
(\coqdocvar{s}, \coqdocvar{g'}) = \coqdocvar{extract\_vertices\_degs} \coqdocvar{g} \coqdocvar{n} \ensuremath{\rightarrow}\coqdoceol
\coqdocindent{2.00em}
(\coqdocvar{s'}, \coqdocvar{g'{}'}) = \coqdocvar{extract\_vertices\_degs} \coqdocvar{g'} \coqdocvar{m} \ensuremath{\rightarrow}\coqdoceol
\coqdocindent{2.00em}
\coqdocvar{S.Empty} (\coqdocvar{S.inter} \coqdocvar{s} \coqdocvar{s'}).\coqdoceol
\coqdocemptyline
\end{coqdoccode}
